
\section{Appendix}


\subsection{Comparing proposition versus FOL logic}


\begin{table}[h!]
\centering
\small
\begin{tabular}{lcc}
\toprule
\textbf{Dataset} & \textbf{Logic-LM Exec} & \textbf{Logify Exec} \\
\midrule
ProofWriter (prop-like) & 99.0 & 99.4 \\
FOLIO (FOL) & 85.8 & 91.2 \\
\bottomrule
\end{tabular}
\caption{Execution rates. Propositional-like tasks achieve near-perfect translation.}
\label{tab:propositional_vs_FOL}
\end{table}

\subsection{Usage Modes}
\label{sec:usage}

The system supports two primary modes:

\paragraph{Logify mode.}
Create a new logified structure from text:
\begin{verbatim}
python main.py from_text_to_logic --text "document.txt"
\end{verbatim}

\paragraph{Query mode.}
Ask questions, optionally adding new text:
\begin{verbatim}
python main.py query --query "Is X true?"
python main.py query --query "Is X true?" --text "additional.txt"
\end{verbatim}

In query mode with additional text, the system first updates the logified structure (Section~\ref{sec:from_text_to_logic}), then processes the query.
